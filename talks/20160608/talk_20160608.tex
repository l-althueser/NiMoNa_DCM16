\documentclass{beamer}
% Nächstes Auskommentieren um jedes \pause zu 'deaktivieren'!
%\documentclass[handout]{beamer}

\usepackage{talk_BeamerColor}
\usepackage{res/meta/meta}
\usepackage[pantoneblack7,english]{talk_wwustyle_LA}

\usepackage[ngerman]{babel}
\usepackage[utf8]{inputenc}
\usepackage[T1]{fontenc}

% --- Paket um Grafiken im Dokument einbetten zu koennen
\usepackage{graphicx}
\usepackage{caption}
\usepackage{subfigure}
\usepackage{wrapfig}

% --- Pakete fuer mathematischen Textsatz
\usepackage{amsmath}
\usepackage{amssymb}
\usepackage{empheq}
\usepackage{dsfont}		
\usepackage{amstext}
\usepackage{amsfonts}
\usepackage{amsthm}
\usepackage{wasysym}

% --- Paket erweitert deutlich die Verwendung von Farben aus dem Paket 'graphicx'
\usepackage{color}

% --- Paket um Quellcode sauber zu formatieren
\usepackage{listings}

% --- Darstellung von Pseudocode und mehr (algorithmicx packages)
\usepackage{algorithm}
\usepackage{algpseudocode}

%% Physikalisches
\usepackage{nicefrac}
\usepackage{units}
\usepackage{siunitx}
\sisetup{
  inter-unit-product 	=	$\cdot$,
  fraction-function   	= 	\nicefrac,
  load-configurations 	= 	abbreviations,
  per-mode            	= 	fraction,
  separate-uncertainty	=	true,
  output-decimal-marker	=	{.}
  }   
\usepackage{isotope}

%% Aufgabenverwaltung
\usepackage[textwidth=2cm,% Breite der Todo-Eintr�ge
            textsize=footnotesize,% Schriftgr��e der Eintr�ge
            english,% deutsche Beschriftungen
            shadow,% Schlagschatten f�r Boxen (weils so h�bsch ist)
            colorinlistoftodos]{todonotes}% farbige Markierungen f�r unterschiedliche Aufgabentypen
\newcommand{\detail}[1]{\todo[color=Green,inline]{detail: #1}~}% Details k�nnten hinzugef�gt werden
\newcommand{\litcheck}[1]{\todo[color=LightSteelBlue,inline]{refcheck: #1}~}% muss noch einmal �berpr�ft werden
\newcommand{\src}[1][]{\todo[color=Tomato,inline]{reference! #1}~}% Quelle fehlt

% % Zusätzliches:
\usepackage{braket}
\usepackage{epstopdf}
\usepackage{pgfpages}
\usepackage{csquotes}

\newlength{\halftextwidth}
\setlength{\halftextwidth}{\textwidth}
\divide\halftextwidth by 2

% --- Einstellungen

\author{NiMoNa 2016}
\title{Konnektivität im Gehirn}
%\institutelogo{Logo on title frame}
%\institutelogosmall{Logo on other frames}
\subtitle{Lutz Althüser, Tobias Frohoff-Hülsmann, Victor Kärcher,\\ Lukas Splitthoff, Timo Wiedemann}
\date[08.06.2016]{08. Juni, 2016}

% --- Beginn des Dokuments

\begin{document}

\begin{frame}[plain]
	  \maketitle
\end{frame}

\begin{frame}{Überblick}
	  \tableofcontents
\end{frame}

\section{Dynamic Causal Modeling - DCM}
\subsection{Motivation}
\subsection{Modell}

\section{Numersiche Algorithmen}
\subsection{Euler}
\subsection{RK4}

\section{Numerische Experimente}
\subsection{linear}
\subsection{bilinear}

\section{Ausblick}

\begin{frame}{Ganz viel Text}
	Füll mich! :)
\end{frame}

\begin{frame}
	\frametitle{Designfeatures}
	\begin{block}{Hervorhebungen}
	 \textbf{Wenn man Dinge hervorheben möchte nutzt man entweder Fettdruck,} \textit{ kursive Schrift} \alert{ oder das Schlüsselwort "alert"}. Auch "itemize"-Umgebungen werden von der Stilvorlage überschrieben:
	\end{block}
	\pause
	\begin{itemize}
	 \item So wird sichergestellt,
	 \item dass alle Elemente der Präsentation 
	 \item dieselbe Farbe nutzen.
	\end{itemize}
	\begin{alertblock}{Achtung!}
	 Hier kommt Rot ins Spiel!	
	\end{alertblock}
	\begin{exampleblock}{Beispiel}
	 Hier kommt Grün ins Spiel!
	\end{exampleblock}
\end{frame}

\end{document}
